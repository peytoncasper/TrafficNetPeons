This section consists of an assessment of the following six components: scope analysis, research completed/remaining, technical analysis, cost analysis, resource analysis and schedule analysis in an attempt to critically analyze the feasibility of the successful fulfillment of the requirements specified herein.
\subsection{Scope Analysis}
The scope of work for all critical requirements is reasonable, but the prototyping of these by the deadline date may be the most challenging aspect of this project. This team is confident however in their ability to fulfill even this most chellenging element. This assessment is based on the research undergone and an understanding of the project completion requirements, as well as the experience provided this team by Dr. Christopher McMurrough, who has been instrumental in aiding their progress thus far. This team anticipates that three of the most crucial requirements will comprise the bulk of the work scope. These are{ }. Furthermore, it can be confidently stated that the addition of all remaining high priority criteria contributes little to the full scope of work for this project, so the probability of completing these remains high.
\subsection{Research}
With regards to this project, team research has spanned a broad spectrum of topics relative to the work scope associated with the project requirements. Thus far this team has reviewed one closely similar open source project (http://www.sonsoftone.com/?page_id=287) which acomplished, relative to the project requirements stated here, a working full HD IP camera based on Raspberry Pi. In addition to this however, they have seen fit to research several topics not covered in the previous source. These topics include...
\subsubsection{When to Use IR Corrected Lenses}
A white paper provided by PELCO Worldwide Headquarters discusses when and how to effectively use IR corrected lenses for day/night cameras. (http://www.crockettsales.com/assets/productdata/IRCorrectedLensesWhitePaper.pdf)
\subsubsection{Technical Guide for Lenses}
A technical guide for cctv lenses provided by Computar with information about auto v. manual iris, video v. DC drive, C v. CS mounting and IR lense functionality. (https://www.surveillance-video.com/media/lanot/attachments/customimport/VM400-Technical-guide.pdf)
\subsubsection{Remaining Research}
In addition to the sources listed herein, this team has ongoing research efforts dedicated to stepper motor control software and, in the near future, will be expanding their research efforts into higher end camera lenses such as the PELCO 13ZD6X10.
\subsection{Technical Analysis}
The envisioned product is not aimed at delivering state of the art technology to the intended field of deployment. This team has been tasked with the development of the best possible product within a specific range of operating constraints. Therefore, the most technically valuable aspects of this project revolve around a venn diagram consisting of cost effectiveness, power efficiency, and durability. Additionally, this team is in competition with a product already in consideration for deployment, providing them a baseline of technical specifications which to improve upon. Given these factors, producing a competetive product is not only feasible, but highly probable.
The question of industry experience needed in developing products of this nature is void, as this team is comprised of a group of inexperienced developers. To compensate, Dr. McMurrough has provided this team with direction when necessary. Excluding his involvement in the development process however, the technical skills necessary to the successful completion of this project which are beyond the skillset of this team remain undiscovered. Those general technical skills which have been learned and utilized thus far include wireless communication, web interfacing, motor control, and power management.
The above listed skills have been honed by this team largely thanks to some key tools employed in the development of the project prototype. Notably the Raspberry Pi, the Raspberry Pi Camera Module, and the adapters associated with them. The diversity of these tools has effectively allowed this team to conduct tests and learn about the technical nature of their project in rapid succession.  
\subsection{Cost Analysis}
Upon the drafting of this document, the prototype remains unfinalized. However this team has not yet committed even one tenth of the available budget to the work currently undergone. They are, at present, within the last quarter of their alloted development time and therefore can be reasonably expected to finish the project prototype with the remaining budget.
\subsection{Resource Analysis}
As previously stated, this team is comprised of a group of inexperienced developers. As such, their skillset is currently under evaluation based upon the performance of their final product. Many of the skills needed to successfully complete the development of this project are being learned at the very same time as they are being utilized throughout the development process. However, those skills such as programming the Raspberry Pi .dts files have been successfully employed by this team. Therefore it is not unreasonable to expect that any remaining skills necessary to the successful completion of this project but not already employed by this team can and will be honed in order to deliver a complete and professional product.
\subsection{Schedule Analysis}
